\part*{Prologue}

\raggedright
% \RaggedRight

Cette nouvelle prend place dans les années 1930 en Allemagne.
Cette période historique est particulièrement dense en informations.
Il est donc nécessaire d'amener quelques précisions historiques pour que chaque lecteur puisse lire sans confusion ce récit.

L'étude de l'Allemagne entre-deux-guerre est un travail subtil, 
en effet la conclusion de celle-ci est l'avènement d'Adolf Hitler au pouvoir d'une puissance majeur européenne.
Il est donc naturel de vouloir trouver une cause unique, simple et fataliste pour ce tragique dénouement.
Le plus souvent, gravitant autour de la crise économique sans précédant que l'Allemagne a subi. 
Ici, nous ne ferons que peu mention du caractère économique de la République de Weimar, au vu du format réduit de l'exercice.

Nous nous concentrerons, au contraire, sur l'aspect politique.\\
Commençons par définir les forces en jeu :

\begin{itemize}
    \item N.S.D.A.P\footnote{\textit{Nationalsozialistische Deutsche Arbeiter Partei}} : Partie national-socialiste des travailleurs allemands.
    \item K.D.P\footnote{\textit{Kommunistische Partei Deutschlands}} : Partie communiste allemand.
\end{itemize}

Nous nous limitons aux deux partis extrêmes pour les raisons énoncées au-dessus.\\
De plus il est important de noter les résultats des élections législatives du 14 septembre 1930 :
18.3 \% pour le N.S.D.A.P et 14.3 \% K.D.P

Avec le parti le plus populaire étant le S.P.D\footnote{\textit{Sozialdemokratische Partei Deutschlands}} faisant un score de 24.5\%, son plus bas depuis le debut de la république.
% Nous nous devons aussi de rappeler la haine bilatérale entre le SPD et le KPD, originant principalement de la conception radicalement différente du moyen de recuperation du pouvoir.
% Leur rivalité a laissé un vide que les nationalistes ont pu remplir.

Nous ferons mention de la police berlinoise et du ministère de l'Intérieur allemand, 
en particulier la \textit{Preußische Geheimpolizei}\footnote{Police secrète prussienne}.
Cette organisation avait originalement pour but de prévenir les révoltes politiques comme celle de 1848.
Elle sera, plus tard, fusionnée avec la police berlinoise.\\ 

C'est finalement à la chute du second empire qu'une réorganisation eu lieu. 
Aboutissant en deux entités, le \textit{Abteilung I der Verwaltungspolizei}\footnote{Département 1 de la police administrative}
rattachée à la police berlinoise, subordonné au \textit{politische gruppe}\footnote{Groupe politique} du \textit{Abteilung II für Polizeiangelegenheiten}\footnote{2$^e$ département des affaires policières du ministère de l'Intérieur}.

1932 marque la fin de la \enquote{police républicaine}. Conséquence directe du coup d'État de Prusse, 
qui laisse la porte ouverte à l'infiltration du département Un par les nationaux-socialistes. 
Cela mène à l'incapacité de la police à gérer le problème national-socialiste.

C'est de cette base que le récit va évoluer.\\
\textit{Nous n'utiliserons aucun nom de personnage historique.}