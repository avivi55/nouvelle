\lettrine{E}{n} ce jour d'automne 1931, la pluie coulait sur le cimetière militaire Berlinois,
le flot assourdissant de l'eau masquait les récitations religieuses.
En face du clerc, notre jeune officier, le regard vide, l'âme blessée, observait, l'orée de la forêt. 
De cette dense broussaille, un courageux renardeau sorti.
De cette scène, on entendait les cries d'une mère qui avait perdu son petit.

Un premier coup de fusils retentit.
L'esprit anesthésié, Jank contemple le ciel, il y vit deux oiseaux se faire face, puis, le rapace aux plumes rouge sang se heurta au grand corbeau noir comme les abysses.
C'était un combat qui ne pouvait que mal se terminer.

Un deuxième coup se fait entendre.
Les deux volatiles s'étaient maintenant engagés dans une danse mortuaire transperçant le ciel.
Ce couple d'ennemies, sans se préoccuper du monde qui les entouraient, finirent leur chute sur le renardeau perdu. 

Le dernier tir résonna, concluant la cérémonie policière.
Sans avoir le temps de cligner des yeux Jank découvrait la scène meurtrière.
Seul l'oiseau funèbre en sortait survivant.

\bigskip

L'ambiance était lourde au commissariat, personne ne s'y attendait.
La mort de Ferdinant Ozil était une nouvelle particulièrement dure à digérer pour notre ami en deuil.

\enquote{%
    On peut pas laisser cette accident nous détourner de nos affaires! 

    Ferdinant Ozil, brave officer ayant servi sa nation, est mort jeudi dernier dans l'explosion qui a eu lieu au siège du KPD.
    Il faut donc réussir à calmer tous le monde et faire en sorte que personne ne commence une révolte.
} Contait le commissaire.
\smallskip

Jank Schmidt ne pouvait accepter que son frère d'arme se fasse oublier ainsi.
C'est pourquoi, au terme de la réunion, Jank, tourmenté d'interrogation, se jeta dans le bureau du commissaire.
En ouvrant la porte il s'exclame
\enquote{Comment une explosion dans le siège d'un des parties politiques les plus controversé du pays puisse n'être qu'un accident ?}.
Il continua
\enquote{Et es-ce une coincidence si l'explosion a eu lieu je jour où Rudolf Krankenwagen, le président du parti, devait y faire sa visite ?}

Le gradé ne savait trouver les mots pour exprimer sa peine, mais, seul l'innaction pouvait préserver la paix.

\smallskip

Notre jeune héro, assoiffé de justice se mit à enquêter sur les veritables causes de la mort de Ferdinant.



