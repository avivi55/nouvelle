\lettrine{E}{n} ce jour d'automne 1931, la pluie coulait sur le cimetière militaire Berlinois,
le flot assourdissant de l'eau masquait les récitations religieuses.
En face du clerc, notre jeune officier, le regard vide, l'âme blessée, observait, l'orée de la forêt. 
De cette dense broussaille, un courageux renardeau sorti.
De cette scène, on entendait les cries d'une mère qui avait perdu son petit.

Un premier coup de fusils retentit.
L'esprit anesthésié, Jank contemple le ciel, il y vit deux oiseaux se faire face, puis, le rapace aux plumes rouge sang se heurta au grand corbeau noir comme les abysses.
C'était un combat qui ne pouvait que mal se terminer.\\
Un deuxième coup se fait entendre.
Les deux volatiles s'étaient maintenant engagés dans une danse mortuaire transperçant le ciel.
Ce couple d'ennemies, sans se préoccuper du monde qui les entouraient, finirent leur chute sur le renardeau perdu.\\
Le dernier tir résonna, concluant la cérémonie policière.
Sans avoir le temps de cligner des yeux Jank découvrait la scène meurtrière.
Seul l'oiseau funèbre en sortait survivant.

L'ambiance était lourde au commissariat, personne ne s'y attendait.
La mort de Ferdinand Ozil était une nouvelle particulièrement dure à digérer pour notre ami en deuil.\\
\enquote{%
    On peut pas laisser cet accident nous détourner de nos affaires !\\
    Ferdinand Ozil, brave officier ayant servi sa nation, est mort jeudi dernier dans l'explosion qui a eu lieu au siège du KPD.
    Il faut donc réussir à calmer tout le monde et faire en sorte que personne ne commence une révolte.}
Contait le brigadier.\\
Jank Schmidt ne pouvait accepter que son frère d'arme se fasse oublier ainsi.
C'est pourquoi, au terme de la réunion, Jank, tourmenté d'interrogation, se jeta dans le bureau du brigadier.
En ouvrant la porte, il s'exclama
\enquote{Comment une explosion dans le siège d'un des parties politiques les plus controversés du pays puisse n'être qu'un accident ?}.
Il continua,
\enquote{Et est-ce une coïncidence si l'explosion a eu lieu le jour où Rudolf Krankenwagen, le président du parti, devait y faire sa visite ?}.
Le gradé ne savait trouver les mots pour exprimer sa peine, mais, seul l'inaction pouvait préserver la paix.

Notre jeune héro, assoiffé de justice se mit à enquêter sur les véritables causes de la mort de Ferdinand.

Pendant trois longs mois, ces questions intraitables le travaillaient jusqu'a la moelle.
Il ne pouvait plus vivre, cette obsession grandissait en lui tel un virus.
Cette maladie s'emparait de lui, il perdit la raison, seul ses visites hebdomadaires dans le sépulcre le consolaient.
Il y passait des relevées à parler des amorces de l'enquête qui l'enivrait tant.

Que ne savait-il pas ? Quelle était la pièce manquante ?

\enquote{Mais je sais déjà tout !} cria-t-il à la tombe inerte.\\
\enquote{Je sais que les fascistes ont fait le coup. Je sais qu'il voulait assassiner le KDP.
Mais pourquoi toi, il savait que Krankenwagen et ses camardes avaient eux un délai, il savait que tu serais le seul ?
Tu étais si près du but, à deux doigts de prouver les sales magouilles des fascistes.
On était les seuls à savoir que tu t'y rendais ce maudit jeudi.} En prononçant ces mots, notre enquêteur compris qui était la véritable cible de l'assassinat.
Ce désastre si bien exécuté, était un moyen si simple de se détourner de la vérité.

Cette révélation emplit Jank d'espoir, mais qui était la taupe ? Mais notre inspecteur n'avait aucunes preuves. 

Le voilà de nouveau vide d'ambition, ce trimestre était donc vain. 
Il décida de suivre l'avis de ses collègues et d'abandonné les recherches et marcha vers sa voiture.
% Le visage abattu, les sourcils froncés, il réajusta son long manteau et marcha vers sa voiture.     sans se soucier du monde qui l'entourait
Il prit la route vers l'alcool, il n'avait d'autre choix que de ce faire oublier cette amère décision.

Il arriva devant le bar, un grand panneau brillait dans la brune, criant \enquote{{\tgothfamily Das Pint}}.
Il y entra, marcha vers le comptoir et commanda un schnaps, puis un deuxième et un troisième.
Ce mécanisme destructeur se répéta jusqu'a l'ébriété. 
Notre ivrogne tourna la tête et croisa le regard avec Rudolf Fritzen, un autre inspecteur, lui aussi venu pour se libérer d'une lourde journée.
La discutions était entamé, les deux confrères sous le joug le l'éthanol se libéraient petit à petit. 
Les paroles coulaient dans un torrent de confession, Jank qui n'avait jusque-là parler à personne, se livrait enfin 
sur la douleur qui le rongeait quant à la mort de son feu protégé.
L'échange s'éternisait, ils ne pouvaient arrêter de fraterniser sur la dur vie qu'ils menaient.
Une ambiance lourde c'était emparée de la pièce, la lumière orangée des réverbères les illuminait. 
Tandis que les hommes ne s'écoutaient qu'à moitié, ils faisaient persister le dialogue.

--- Je te le dis, je vais retrouver le salop qui à tuer Rudolf.\\
--- De toute façon qui te dit que c'est le NSDAP ?\\
--- C'est forcément leur faute ! Ça ne peut pas être des coïncidences.\\
--- En même temps il va enquêter sur le NSDAP au QG du KDP, c'est une occasion parfaite, dit-il en rigolant.

Cette parlote prit fin sans plus de philosophie.
Jank réussi tant bien que mal à tituber jusqu'à son lit douillé, où il tomba dans les bras de Morphée.

Le lendemain, un éclairé samedi matin, il se réveilla avec beaucoup de difficulté.
Pour se revigorer il plongea la tête dans une bassine d'eau. 
Dans ce moment précis, une décharge électrique lui traversa le corps, il se souvenait de la palabre interminable.
En particulier l'ultime commentaire de Rudolf Fritzen, celui-ci ne pouvait pas avoir connaissance de l'enquête mener par Ferdinand.
Les gouttes froides tombant du visage de Jank rythmaient ses pensés, le doute s'intensifiait et Rudolf Fritzen devint le suspect primaire.

Jank était convaincu, il devait récolter les preuves. 
Il reprit son quotidien et réintégra ses anciennes missions.
Pendant des semaines durant, à chaque occasion qui se présentait à lui, appareil photo Leica dans son manteau, il suivait Fritzen.
Mais rien de très concluant ne liait le suspect au meurtre de l'inspecteur Ozil.

Jank ne pouvait plus attendre, voila maintenant quatre mois que son ami était mort.
Justice n'était pas rendue, en ignorant un tel homicide, l'aveuglement de celle-ci outrepassait l'entendement.

Il se devait d'agir, il se rendit chez le procureur, mais les routes étaient pleines.
Un chahut grondait au loin, les voitures s'empilaient les unes derrières les autres.
Jank sorti de sa voiture, et devant ses yeux se dressa une foule d'hommes costumés en beige, portant le symbole de l'éternité.
Il vit la section d'assaut du NSDAP faire une marche de revendication.
Jank rangea sa voiture aussi vit qu'il pût, pour prévenir de ce rassemblement aux forces de l'ordre.
Il courait sans relâche vers le poste, la respiration difficile, la sueur occupant son front, il tourna la tête, et, comme au bar vit Fritzen.
En regardant de plus près, le vilain personnage serrait la main de Langston Benz, le dirigeant du NSDAP Berlin.
La voilà la preuve qu'il attendait tant, sans hésiter il dégaina son appareil photo et captura cette image autant que possible.

Sans avoir le temps de réfléchir, il courut pour faire développer les photos, reposant toute sa confiance dans la république pour en finir avec cette histoire.
Une fois finit, il se précipita vers de bureau du procureur et du brigadier pour leur montrer les preuves irréfutables.
Mais à sa grande désillusion, ils le regardaient avec incompréhension lui affirmant que la probabilité qu'un officier soit le criminel était faible.
\enquote{De toute façon, il rentre à Munich là.} ricanait le brigadier. 
Digne, Jank s'excusa hors du bureau. 
Les hommes venaient de détruire l'once de bonté qui raisonnait encore en lui.

Dans sa rage de revanche, il se dirigea vers la gare où il prit, sans considération, le premier train pour Munich.
Il fouillait chaque cabine pour le perfide homme.
Il ouvrit la porte d'une voiture, et, il le vit, l'homme qui à engendrer la mort de Ferdinand.

